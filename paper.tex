\documentclass[11pt]{article}
\usepackage[margin=1in]{geometry}
\usepackage[utf8]{inputenc}
\usepackage[dvipsnames]{xcolor}
\usepackage{listings}
\lstset{language=Python,
        tabsize=2,
        numbers=left,
        numberstyle=\tiny,
        numbersep=5pt,
        breaklines=true,
        showstringspaces=false,
        escapeinside={(*@}{@*)}, 
        basicstyle=\small,
        identifierstyle=\color{Black},
        keywordstyle=\color{BurntOrange}\ttfamily,
        morekeywords={True, False},
        stringstyle=\color{red}\ttfamily,
        commentstyle=\color{BrickRed}\ttfamily
}
\lstdefinestyle{inline}{
    columns=fullflexible,
    breaklines=false
}
\newcommand{\code}[1]{\lstinline[style=inline]!#1!}
\newcommand{\blue}[1]{\textcolor{blue}{#1}}
 
\title{Design and Implementation of a Confidential Dropbox Client}

\author{
  Muggler, Michael\\
  \texttt{mxm121531@utdallas.edu}
  \and
  Nathan, Ritchey\\
  \texttt{nar140730@utdallas.edu} 
  \and
  Stephen, Balhoff\\
  \texttt{sjb091020@utdallas.edu}
}

\date{November 2015}

\begin{document}

\maketitle

\section{Introduction}

The goal of this project is to develop an application that adds another level of encryption to Dropbox. The attack model is confidentiality loss once data is uploaded to Dropbox. Using the Dropbox Public REST API we will encrypt data to be uploaded, this includes file and folder metadata. To save on performance files will be divided into blocks, each of which are encrypted separately. After file modification only affected blocks will be altered. The motivation of this project is to maintain confidentiality from Dropbox, active adversaries on the line, and\/or government entities.


% \section{Previous Work}
% Currently excluding previous work, not mentioned in project requirments

% \section{Motivation}
% Combining Motivation with Introduction




\section{Implementation: Major Steps}
\subsection{Block Encryption: Algorithm}

\begin{lstlisting}
tmp_blocks     = new_file # for simplicity, assume new_file is string
cur_sequence   = 0  	  # initialization
def add_blocks_to_DropBox(blocks_to_add):#(*@ \blue{(1)} @*)loop through each block to add
	for new_block in blocks_to_add: 
		++cur_sequence
		add_block_DropBox(new_block, UUID(new_block)) 
		add_index(cur_sequence, UUID(new_block), metadata_UUID)
for block in old_blocks: # (*@ \blue{(2)} @*) the below loop looks at each block of old file
	if block not in tmp_blocks: # (*@ \blue{(3)} @*) block has been removed/edited
		remove_block_from_DropBox(UUID(block))
		remove_block_from_index(UUID(block))
		continue
	# (*@ \blue{(4)} @*) below split will return a list [blocks_edited, old_block, blocks_rest]
	tmp_blocks_split = split(block, tmp_blocks)
	blocks_to_add = break_blocks_on_byte_size(blocks_edited)#(5)
	add_blocks_to_DropBox(blocks_to_add)
	++cur_sequence # (*@ \blue{(6)} @*)
	change_index(cur_sequence, UUID(block)) #change index of the old_block 
	tmp_blocks = blocks_rest # (*@ \blue{(7)} @*)
blocks_to_add = break_blocks_on_byte_size(tmp_blocks) # (*@ \blue{(8)} @*)
add_blocks_to_DropBox(blocks_to_add) #adds rest of blocks
\end{lstlisting}   

\section{Verification}

\subsection{Block Encryption Algorithm: Correctness}

To prove correctness we must show: 
\begin{description}
\item[\#1] All additions, deletions, and edits from the new file must be accounted for. 
\item[\#2] All unedited blocks of old file remain unaltered, (i.e are not updated).
\item[\#3] Unedited blocks update file index for changes of the block index number (increase/decrease). \item[\#4] Edited blocks have the correct corresponding block sequence number represented in the index.
\end{description}

\subsection{Block Encryption Algorithm: Proof of correctness for \#1}

\begin{itemize}
\item Every missing old block from \blue{(2)} is removed through \blue{(3)}. 

\begin{description} 
\item[Therefore:] all deletions are taken into account for the new file.
\end{description}

\item For every old block from \blue{(2)} all blocks of the new file previous to the location of the old block will be added to DropBox. \blue{(8)} will add the last edited set of blocks. This is true because tmp\_blocks will be either all blocks of the new\_file, or all blocks after the previously matched old\_blocks. This includes all added blocks. 

\begin{description}
\item[Therefore:] all additions are taken into account for the new file.
\end{description}

\item Each old block from \blue{(2)}, or at \blue{(8)}. All edits are treated as deletions from the old\_file and additions towards the new file. This is true for deletions because an edited block will not match for any step in \blue{(2)}, thus will be deleted from dropbox. Since the edited block is not included in removed blocks, it has to be a part of either the edited\_blocks in \blue{(5)} at some point in the algorithm or in tmp\_blocks in \blue{(8)}. 

\begin{description}
\item[Therefore:] all edits are taken into account for the new file.
\end{description}

\item All deletions, additions, and edits are taken into account for the file. 

\begin{description}
\item[Thus: \#1] is true.
\end{description}

\end{itemize}

\subsection{Block Encryption Algorithm: Proof of correctness for \#2}

\begin{itemize}
\item Each old block from \blue{(2)}. \blue{(4)} separates the found old\_block from all data previous to location of old\_block (i.e. blocks\_edited) and all data following the end of the old\_block (i.e. blocks\_rest). Since the previous iteration would have caught earlier matching old\_blocks we know that blocks\_edited is free of matching old\_blocks, making \blue{(5)} safe. By definition, if all old\_blocks have been iterated through there must be no old\_blocks in in blocks\_rest\/tmp\_blocks, making \blue{(8)} safe.Therefore all unedited blocks remain unchanged since all block additions, i.e. \blue{(5)} and \blue{(8)}, are safe and do not equal any old blocks, and all block deletions \blue{(3)} only happens when an old\_block doesn't match.
\begin{description} 
\item[Thus: \#2] is true.
\end{description}
\end{itemize}

\subsection{Block Encryption Algorithm: Proof correctness \#3 \& \#4}

\begin{itemize}
\item  Each old block from \blue{(2)}. The updated index consists all blocks generated at each step since all unedited blocks must be matched by a corresponding old\_block and undergo \blue{(6)}, and every new\_block(s) added must complete \blue{(1)}. At each point \blue{(6)} and \blue{(1)} cur\_sequence is incremented by one.This is done in order as through each iteration of all old\_blocks old\_block(n) must be added before old\_block(n+1). \blue{(2)} updates index of all edited blocks previous to index update of the matched old\_block. This is done in order since the addition of blocks is done linearly. \blue{(8)} adds index's of all remaining new edited blocks in order. Since the new file is the summation of all new blocks and matching old blocks, and these blocks are inserted in order linearly, that means each block has the correct corresponding index number.
\begin{description}
\item[Thus: \#3 \& \#4] are true.
\end{description}
\end{itemize}

\section{Difficulties: During Implementation}

\subsection{Problem: 1}
If you split file X into blocks, assign each block a UUID. These blocks will have min size of bytes which will be the key size. The last block will have the extra bytes [Size(X) \% Size(key)], so it will be Size(key) + (Size(X) \% Size(key)). Each block is stored on DropBox as the pair <encrypted block, authenticated hash(block)>. Then there is an index that stores the order of these files. The issue lies where in if you edit block(x), and add data that increases the size such that (Size(d) \% Size(key)) != 0, then what will happen is that the extra bytes will overflow into the next block and in turn could possibly affect all subsequent blocks. This means that a user will encrypt more blocks than necessary, and overwrite blocks that they did not change. This violates two goals of the project.

\subsection{Solution: 1}
See "Block Encryption Algorithm" under "Implementation".

\section{Division of Labor}

\subsection{Muggler, Michael}
\begin{itemize}
\item Being God King of Pluto
\end{itemize}

\subsection{Nathan, Ritchey}
\begin{itemize}
    \item Designing the Block Encryption Algorithm.
    \item Implementing multiple client capability. TBI
    \item Heavy Report Creation.
\end{itemize}

\subsection{Stephen, Balhoff}
\begin{itemize}
    \item Doot Doot Doot
    \item Performing duties as Future Time Cop from the 1940's
\end{itemize}

%\section{Conclusion}
%Removing Conclusion for now, not in requirments

\end{document}
